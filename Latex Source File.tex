\documentclass{article}
\usepackage[margin=1in]{geometry} % Adjust margins as needed
\usepackage{graphicx}
\usepackage{hyperref}
\title{Solar System Project}

\begin{document}
\maketitle

\section{Introduction}
In this project, we explore the representation of the solar system in a web page. The web page uses HTML, CSS, and JavaScript to create a visual representation of the solar system, complete with planets and links to their information.

\section{HTML and CSS}
The web page is structured using HTML and styled with CSS. It includes a background image, images of the Sun and planets, and elements for displaying planet names and links. The use of CSS ensures proper positioning and styling of the elements.

\section{JavaScript Functionality}
JavaScript is used to add interactivity to the web page. Each planet button is associated with a specific planet's information link. When a planet button is clicked, the JavaScript code opens a new tab with the corresponding link. This enhances the user experience by providing information about each planet.

\section{Debugging and Profiling}
To ensure the functionality and performance of the web page, debugging and profiling techniques are used. Browser developer tools are employed to identify and rectify issues in the JavaScript code. Additionally, profiling is performed to analyze the performance of the page and optimize it for better user experience.

\section{Conclusion}
The Solar System project showcases the use of web technologies to create an interactive and informative web page. It offers a glimpse into the structure of the solar system and allows users to access detailed information about each planet.

\section{References}
- [ThoughtCo - Basic Information About the Sun](https://www.thoughtco.com/basic-information-about-the-sun-3073700)
- [Wikipedia - Mercury (planet)](https://en.wikipedia.org/wiki/Mercury_(planet))
- [Encyclopedia Britannica - Venus (planet)](https://www.britannica.com/place/Venus-planet)
- [Wikipedia - Earth](https://en.wikipedia.org/wiki/Earth)
- [Wikipedia - Mars](https://en.wikipedia.org/wiki/Mars)
- [Wikipedia - Jupiter](https://en.wikipedia.org/wiki/Jupiter)
- [Wikipedia - Saturn](https://en.wikipedia.org/wiki/Saturn)
- [Wikipedia - Uranus](https://en.wikipedia.org/wiki/Uranus)
- [Wikipedia - Neptune](https://en.wikipedia.org/wiki/Neptune)

\end{document}
